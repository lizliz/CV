%%%%%%%%%%%%%%%%%%%%%%%%%%%%%%%%%%%%%%%%%
% Medium Length Professional CV
% LaTeX Template
% Version 2.0 (8/5/13)
%
% This template has been downloaded from:
% http://www.LaTeXTemplates.com
%
% Original author:
% Trey Hunner (http://www.treyhunner.com/)
%
% Important note:
% This template requires the resume.cls file to be in the same directory as the
% .tex file. The resume.cls file provides the resume style used for structuring the
% document.
%
%%%%%%%%%%%%%%%%%%%%%%%%%%%%%%%%%%%%%%%%%

%----------------------------------------------------------------------------------------
%	PACKAGES AND OTHER DOCUMENT CONFIGURATIONS
%----------------------------------------------------------------------------------------

\documentclass{resume} % Use the custom resume.cls style



%-----------Mess with column widths for teaching table----------------%
\newcolumntype{L}{>{\centering\arraybackslash}m{6.5cm}}



%------------bibliography-------------------$
\usepackage[backend=bibtex, sorting = ydnt, defernumbers=true, maxbibnames=99, url=false]{biblatex}
\bibliography{MyPublications} % or
%  \usepackage[style=authoryear, backend=bibtex]{biblatex}
% \addbibresource{MyPublications.bib}


%-------Remove the reference title from publications section----------%
\usepackage{etoolbox}
\patchcmd{\thebibliography}{\section*{\refname}}{}{}{}


\usepackage{amsfonts}

\usepackage[left=0.75in,top=0.6in,right=0.75in,bottom=0.6in]{geometry} % Document margins
\usepackage{colortbl}
\usepackage[usenames,dvipsnames]{xcolor}


\usepackage{etaremune}
\usepackage{color,hyperref}
\definecolor{darkblue}{rgb}{0.0,0.0,0.3}
\hypersetup{colorlinks,breaklinks,
            linkcolor=darkblue,urlcolor=darkblue,
            anchorcolor=darkblue,citecolor=darkblue}

\usepackage[symbol]{footmisc}

\renewcommand{\thefootnote}{\fnsymbol{footnote}}
% \footnote[num]{text}.... see https://tex.stackexchange.com/questions/826/symbols-instead-of-numbers-as-footnote-markers

\name{Elizabeth Munch} % Your name
\address{Dept of Computational Mathematics, Science, and Engineering \\ Dept of Mathematics}
\address{Michigan State University \\ East Lansing, MI} % Your address
\address{517-432-0619 \\ muncheli@msu.edu} % Your phone number and email

\begin{document}

%----------------------------------------------------------------------------------------
%	EDUCATION SECTION
%----------------------------------------------------------------------------------------

\begin{rSection}{Education}

\college
{\href{http://www.duke.edu/}{Duke University}}
{Durham, NC}
\begin{DegreeList}{Ph.D.~Dept.~of Mathematics}{May 2013}
%  \item Advisor: John Harer
 \item Thesis: Applications of Persistent Homology to Time Varying Systems
\end{DegreeList}
\Degree
{M.A.~Dept.~of Mathematics}
{Dec 2010}


\college
{\href{http://www.rochester.edu}{\textbf{University of Rochester}}}
{Rochester, NY}
\Degree
{B.S.~Mathematics, Summa Cum Laude, School of Arts and Sciences}
{May 2008}\\
\Degree
{B.M.~Harp Performance with High Distinction,  Eastman School of Music}
{May 2008}





\end{rSection}

%----------------------------------------------------------------------------------------
%	WORK EXPERIENCE SECTION
%----------------------------------------------------------------------------------------

\begin{rSection}{Research Experience}

\college
{\href{http://msu.edu/}{Michigan State University}}
{East Lansing, MI}
Dept.~of Computational Mathematics, Science and Engineering (CMSE)\\
 Dept.~of Mathematics\\
\Degree
{Associate Professor}
{July 2022 -- Present}\\
\Degree
{Assistant Professor}
{Aug 2017 -- July 2022}

\college
{\href{http://www.albany.edu/}{University at Albany -- SUNY}}
{Albany, NY}
 Dept.~of Mathematics \& Statistics\\
\Degree
{Assistant Professor}
{Sept 2014 -- July 2017}\\
 Dept.~of Computer Science\\
\Degree
{Affiliated Faculty}
{July 2015 -- July 2017}

\college
{\href{http://www.umn.edu/}{University of Minnesota}}
{Minneapolis, MN}
Institute for Mathematics and Its Applications\\
\Degree
{Postdoctoral Fellow}
{Sept 2013 -- Aug 2014}

\college
{\href{http://www.duke.edu/}{Duke University}}
{Durham, NC}
Dept.~of Mathematics\\
\Degree
{Visiting Assistant Professor}
{June 2013 -- Aug 2013}\\
\Degree
{Graduate Research Assistant}
{Sept 2008 -- May 2013}


% %------------------------------------------------
%
% \begin{rSubsection}{AJAX Hosting}{December 2009 - October 2010}{Lead Developer}{Austin, TX}
% \item Aenean ut gravida lorem. Ut turpis felis, Perl pulvinar a semper sed, adipiscing id dolor.
% \item Curabitur dapibus enim sit amet elit pharetra tincidunt website feugiat nisl imperdiet. Ut convallis AJAX libero in urna ultrices accumsan.
% \item Cum sociis natoque penatibus et magnis dis MySQL parturient montes, nascetur ridiculus mus.
% \item In rutrum accumsan ultricies. Mauris vitae nisi at sem facilisis semper ac in est.
% \item Nullam cursus suscipit nisi, et ultrices justo sodales nec. Fusce venenatis facilisis lectus ac semper.
% \end{rSubsection}
%
% %------------------------------------------------
%
% \begin{rSubsection}{TinySoft}{January 2008 - April 2010}{Web Designer \& Developer}{Gainesville, GA}
% \item Vivamus PostgreSQL fermentum semper porta. Nunc diam velit PHP, adipiscing ut tristique vitae
% \item Maecenas convallis ullamcorper ultricies stylesheets.
% \item Quisque mi metus, unit tests CSS ornare sit amet fermentum et, tincidunt et orci.
% \item Curabitur venenatis pulvinar tellus gravida ornare. Sed et erat faucibus nunc euismod ultricies ut id
% \end{rSubsection}

\end{rSection}


% \newpage
%----------------------------------------------------------------------------------------
%	Awards and Honors
%----------------------------------------------------------------------------------------

\begin{rSection}{Awards and Honors}

% \begin{list}{}{}
\begin{list}{}{\leftmargin=0em}
   \itemsep -0.6em %\vspace{-0.5em} % Compress items in list together for aesthetics
\item NSF CAREER Award, 2022
\item
Jo Rae Wright Fellowship for Outstanding Women in Science, Duke University, 2012-2013

\item
Phi Beta Kappa, University of Rochester, May 2008


\item
Performer's Certificate in Harp, May 2008



\item
Lois S. Rogers Scholarship, Eastman School of Music, 2004-2008
%Eileen Malone Memorial Scholarship, Eastman School of Music, 2007.\\
%Did I get this one????

\item Performer's Certificate, Eastman School of Music, 2008

\item
Presser Scholarship, Eastman School of Music, 2007


\end{list}


\end{rSection}


%----------------------------------------------------------------------------------------
%	Publications
%----------------------------------------------------------------------------------------
\newpage

\begin{rSection}{Publications}
\textit{Due to working in an interdisciplinary setting, my work follows many different conventions for authorship.
Theoretical mathematics and theoretical computer science tend to be published alphabetically.
Applied mathematics and other domain settings tend to be published in descending order of contribution, with graduate students and postdocs listed first; followed by PIs.
% Symbols used to denote authorship:
}
% \begin{itemize}{}{\leftmargin=0em}
%    \itemsep -0.6em %\vspace{-0.5em} % Compress items in list together for aesthetics
%   \item [$\dagger$] \textit{MSU Graduate Student Advisee}
%   \item [$\ddagger$] \textit{Non-MSU Graduate Student Advisee}
%   \item [$\circ$] \textit{MSU Postdoc Advisee}
%   \item [$\star$] \textit{Other MSU Graduate Students and Postdocs}
%   \item [$\diamond$] \textit{REU Undergraduate}
%
%
%   % \item [$\oplus$] Corresponding author
% \end{itemize}


\nocite{*}

\textbf{Preprints}
% --- If you want all preprints --
% \printbibliography[keyword = preprint, heading=none]
% --- If you only want posted preprints --
% \printbibliography[keyword = arxiv, heading=none]
% --- If you only want in submission preprints --
\printbibliography[keyword = inSubmission, heading=none]


\textbf{Journal Articles}\\
\textit{Whenever possible, I post preprints of my papers to the arXiv. Where available, I include links to both the publised versions (via DOI or URL), as well as to the arXiv posting.}
% \printbibliography[keyword = journal, heading=none, resetnumbers = true]
\printbibliography[keyword = journal, heading=none]


\textbf{Computer Science Conference Proceedings}\\
\textit{My work is interdisciplinary and often crosses between mathematics and theoretical computer science.
In CS, both journals and selective proceedings are highly regarded venues for publications.
The proceedings have the advantage of high visibility and a shortened time to press. These venues are peer-reviewed by a general minimum of between 3 and 5 evaluators and can be highly competitive. Acceptance rates are provided when known. Most notable conferences in my work are the \emph{Symposium on Computational Geometry (SoCG)}, which is the top conference in Applied Topology and Geometry \cite{Chambers2020,Munch2016,Bauer2015b}; and IEEE Scientific Visualization (SciVis) published in the journal TVCG, which is the top conference in visualization \cite{Yan2019a}.}
\printbibliography[keyword = conferenceTop, heading=none]


\textbf{Other Conference Proceedings}

\textit{This section includes conference papers either from non-computer science conferences, or from papers submitted to workshops where the paper is not part of the main session.}

\printbibliography[keyword = conference, heading=none]

\textbf{Book Chapters}
\printbibliography[keyword = book, heading=none]


\textbf{Published Datasets}
\printbibliography[keyword = dataset, heading=none]

\textbf{Other Published Work}
\printbibliography[keyword = other, heading=none]

% \nocite{*}

% \nocite{deSilva2015, Munch2012}
%
% Second
% \nocite{deSilva2015, Munch2012}
% \bibliographystyle{plainyrRev}
% \bibliography{MyPublications}

% \begin{refsection}
% \end{refsection}
% \printbibliography[keyword=journal]
%
% \printbibliography[keyword=conference]
% \printbibliography[keyword=preprint]
\end{rSection}

%----------------------------------------------------------------------------------------
%
%----------------------------------------------------------------------------------------



% \newpage
\begin{rSection}{Grants: Funded and Recommended for Funding}




\textbf{\textit{Collaborative Research: AF: Medium: A Unified Framework for Geometric and Topological Signature-Based Shape Comparison}}
\vspace{-.1in}
\begin{itemize}{}{\leftmargin=0em}
   \itemsep -0.6em %\vspace{-0.5em} % Compress items in list together for aesthetics
   \item
NSF \href{https://nsf.gov/awardsearch/showAward?AWD_ID=2106578}{CCF-2106578}:  \$409,945
\item
Jun 2021 - May 2025
   \item
Role: Principal Investigator
\item Collaborative with
  Erin Chambers, Computer Science, St.~Louis University; Carola Wenk, Computer Science, Tulane University

\end{itemize}

\textbf{\textit{AF: Small: Collaborative Research: Reeb graph flows: Metrics, Drawings, and Analysis}}
\vspace{-.1in}
\begin{itemize}{}{\leftmargin=0em}
   \itemsep -0.6em %\vspace{-0.5em} % Compress items in list together for aesthetics
   \item
NSF \href{https://nsf.gov/awardsearch/showAward?AWD_ID=1907591}{CCF-1907591}:  \$246,596
\item
Oct 2019 - Sept 2022
   \item
Role: Principal Investigator
\item Collaborative with
  Erin Chambers, (PI: CCF-1907612) Computer Science, St.~Louis University

\end{itemize}

%-------------
% \vspace{-.1in}
%-------------
\textbf{\textit{Role of transport processes in formation of optimal microbial habitats and the root-microbe-soil carbon accrual continuum}}
\vspace{-.1in}
\begin{itemize}{}{\leftmargin=0em}
   \itemsep -0.6em %\vspace{-0.5em} % Compress items in list together for aesthetics
   \item
NSF DEB-\href{https://nsf.gov/awardsearch/showAward?AWD_ID=1904267}{1904267}: \$624,995
\item
Sept 2019 - Aug 2024
   \item
Role: Co-PI
\item Collaborative with
   \vspace{-.1in}
\begin{itemize}{}{\leftmargin=0em}
   \itemsep -0.6em %\vspace{-0.5em} % Compress items in list together for aesthetics
  \item Alexandra Kravchenko (PI),  Dept of Plant, Soil, and Microbial Sciences, MSU
  \item Daniel Chitwood (Co-PI), Depts of Horticulture and CMSE, MSU
  \item Lisa Tiemann (Co-PI), Dept of Plant, Soil, and Microbial Sciences, MSU
  \item Andrey Guber (Co-PI), Dept of Plant, Soil, and Microbial Sciences, MSU
\end{itemize}
\end{itemize}
%-------
\textbf{\textit{Zigzag Persistent Homology and Network Methods for Topological Signal Processing}}
\vspace{-.1in}
\begin{itemize}{}{\leftmargin=0em}
   \itemsep -0.6em %\vspace{-0.5em} % Compress items in list together for aesthetics
   \item Air Force Office of Scientific Research, \$375,000
\item
Oct 2021 - Sep 2024
   \item
Role: Co-PI
\item Collaborative with Firas Khasawneh (PI), Mechanical Engineering, MSU

\end{itemize}
%-------
\textbf{\textit{CAREER: Reeb graph learning: Classification, Clustering, and Embedding of Graphical Signatures
}}
\vspace{-.1in}
\begin{itemize}{}{\leftmargin=0em}
   \itemsep -0.6em %\vspace{-0.5em} % Compress items in list together for aesthetics
   \item NSF CCF-\href{https://nsf.gov/awardsearch/showAward?AWD_ID=2142713}{2142713}:  \$507,462
\item
May 2022 - May 2027
   \item
Role: PI

\end{itemize}



%-------------
% \vspace{-.1in}
%-------------
%=============
\end{rSection}
\begin{rSection}{Grants: Completed Projects}
%=============
\textbf{\textit{Collaborative Research: A Unified Framework for the Investigation of Time Series Using Topological Data Analysis}}
\vspace{-.1in}
\begin{itemize}{}{\leftmargin=0em}
   \itemsep -0.6em %\vspace{-0.5em} % Compress items in list together for aesthetics
   \item
NSF CMMI-\href{https://nsf.gov/awardsearch/showAward?AWD_ID=1562012}{1562012}/\href{https://nsf.gov/awardsearch/showAward?AWD_ID=1800466}{1800466}: (UAlbany/MSU)  \$178,736 \footnote{Double grant numbers caused by transfer of grants to MSU.}
\item
April 2016 - Mar 2020
   \item
Role: Principal Investigator
\item Collaborative with
  Firas Khasawneh, (PI: CMMI-1562459/1759823) Mechanical Eng., MSU
%    \item
% NSF CMMI-1562459/1759823: (SUNY Poly/MSU) \$196,499 (PI Khasawneh)

\end{itemize}

%-------------
% \vspace{-.1in}
%-------------

\textbf{\textit{CDS\&E: Collaborative Research: Machine Learning on Dynamical Systems\\ via Topological Features}}
\vspace{-.1in}
\begin{itemize}{}{\leftmargin=0em}
   \itemsep -0.6em %\vspace{-0.5em} % Compress items in list together for aesthetics
   \item
NSF DMS-\href{https://nsf.gov/awardsearch/showAward?AWD_ID=1622320}{1622320}/\href{https://nsf.gov/awardsearch/showAward?AWD_ID=1800446}{1800446}: (UAlbany/MSU)  \$101,672
\item
Sept 2016 - Aug 2019
   \item
Role: Principal Investigator
\item Collaborative with
   \vspace{-.1in}
\begin{itemize}{}{\leftmargin=0em}
   \itemsep -0.6em %\vspace{-0.5em} % Compress items in list together for aesthetics
   \item Firas Khasawneh, (PI: DMS-1622293/1759824) Mechanical Eng., MSU
   \item Jos\'e Perea, (PI: DMS-1622301) Depts.~of Math, \& CMSE, MSU
   \end{itemize}
   % \item
% NSF DMS-1622301: (MSU)  \$105,000 (PI Perea)
   % \item
% NSF DMS-1622293/1759824: (SUNY Poly/MSU)  \$93,111 (PI Khasawneh)


\end{itemize}



% \newpage

\textbf{\textit{Kaleidoscope: Turning System Design Inside-Out}}
\vspace{-.1in}
\begin{itemize}{}{\leftmargin=0em}
   \itemsep -0.6em %\vspace{-0.5em} % Compress items in list together for aesthetics
   \item
DARPA
\item
Aug 2016-July 2017
   \item
Role: Technical POC
\item Collaborative with
Raytheon BBN Technologies Corp., Boston, MA
   % \item
% Total award amount: \$308,089
% \vspace{-.1in}
% \begin{itemize}
% {}{\leftmargin=0em}
%    \itemsep -0.6em %\vspace{-0.5em} % Compress items in list together for aesthetics
%    \item
% NSF DMS-1622320: (UAlbany)  \$101,672 (PI Munch)
%    \item
% NSF DMS-1622301: (MSU)  \$105,000 (PI Perea)
%    \item
% NSF DMS-1622293: (SUNY Poly)  \$93,111 (PI Khasawneh)

% \end{itemize}

\end{itemize}
\vspace{-.1in}

\end{rSection}


%
% ----------------------------------------------------------------------------------------
% \newpage
\begin{rSection}{Invited Speaker}
\noindent{Talk titles with hyperlinks have recordings or supplementary  material available.}

\begin{etaremune}
\item\emph{The Directional Transform, or how to look at your data from every direction at once}. Data-oriented Mathematical \& Statistical Sciences (DoMSS) Seminar, Arizona State University, Virtual, Oct 17, 2022. 

\item\emph{\href{https://www.youtube.com/watch?v=_-gDBosxp2M}{The Directional Transform, or how to look at your data from every direction at once}}. AATRN STMS Joint Seminar Series, Virtual, Oct 14, 2022. 

\item\emph{Graph Classification Using the Interleaving Distance}. SIAM-MDS, San Diego, CA, Sep 29, 2022. 

\item\emph{Crafting Topological Features}. BIRS Deep Exploration of non-Euclidean Data with Geometric and Topological Representation Learning, Kelowna, BC, Canada, July 12, 2022. 

\item\emph{The many faces of the interleaving distance}. Plenary talk, ATMCS 10, Oxford, UK, June 23, 2022. 

\item\emph{Making space and studying shape in a non-traditional academic setting}. Keynote Address, WiDS Iowa, Virtual, Apr 21, 2022. 

\item\emph{The Interleaving Distance for Reeb Graphs}. Math Dept Colloquium, University of Utah, Salt Lake City, UT, Apr 7, 2022. 

\item\emph{The Shape of Data}. Keynote Address, NAPPN Annual Conference, University of Georgia, Athens, GA, Twosday, Feb 22, 2022. 

\item\emph{Combining network analysis and persistent homology for classifying behavior of time series}. Dynamics Days, Online due to COVID-19, Jan 7, 2022. 

\item\emph{The Truncated Interleaving Distance for Reeb Graphs}. Quivers Seminar, University of Iowa, Online due to COVID-19, November 15, 2021. 

\item\emph{\href{https://drive.google.com/file/d/1lnZ_g1aQBkX4W6F9SRovgQ8Bzih1Is97/view?usp=sharing}{On the inter-level-set persistence bottleneck distance for Reeb graphs}}. Computational Persistence Workshop, Purdue, Online due to COVID-19, November 5, 2021. 

\item\emph{The Truncated Interleaving Distance for Reeb Graphs}. SIAM Conference on Applied Algebraic Geometry, Online due to COVID-19, August 18, 2021. 

\item\emph{The Interleaving Distance for Graphical Signatures}. Mathematical Congress of the Americas, Online due to COVID-19, July 16, 2021. 

\item\emph{Measuring the Shape of Hurricanes}. MSU SURIEM REU, Online due to COVID-19, July 7, 2021. 

\item\emph{Metrics for Graphical Signatures}. Topological Ideas in Applications, Thematic Einstein Semester on Geometric and Topological Structure of Materials, TU Berlin, Online due to COVID-19, June 24, 2021. 

\item\emph{The Directional Transform: From Theory to Practice}. Workshop on Computational Topology at SoCG, Gather town, June 10, 2021. 

\item\emph{A family of metrics from the truncated smoothing of Reeb graphs}. Symposium on Computational Geometry (SoCG) Main Session, Gather town, June 7, 2021. 

\item\emph{Quantifying the shape of time series with TDA and network-based methods}. AI and Topology Track, EPFL Applied Machine Learning Days, Online due to COVID-19, May 10, 2021. 

\item\emph{\href{https://www.youtube.com/watch?v=EUVk-FXbIog}{Combining network analysis and persistent homology for classifying behavior of time series}}. Applied Mathematics and Complex Systems Seminar, University of Western Australia, Online due to COVID-19, May 6, 2021. 

\item\emph{The Truncated Interleaving Distance for Reeb Graphs}. Topological Data Analysis Workshop, IMSI, Online due to COVID-19, April 27, 2021. 

\item\emph{Utilizing Persistent Homology for Classifying Behavior of Time Series}. Computational Topology Class, University of Utah, Online due to COVID-19, April 22, 2021. 

\item\emph{Measuring the Shape of Hurricanes}. MSU Undergraduate Math Club, Online due to COVID-19, April 6, 2021. 

\item\emph{Combining network analysis and persistent homology for classifying behavior of time series}. Applied Math Colloquium, University of Arizona, Online due to COVID-19, March 12, 2021. 

\item\emph{Averaging Merge Trees}. EPFL Applied Topology Seminar, Online due to COVID-19, March 9, 2021. 

\item\emph{The Truncated Interleaving Distance for Reeb Graphs}. Joint Mathematics Meetings, Online due to COVID-19, Jan 9, 2021. 

\item\emph{\href{https://github.com/lizliz/TDA-Python-Workshop-JMM21}{Python Tutorial on Topological Data Analysis (TDA)}}. AMS Short Course on Mathematical and Computational Methods for Complex Social Systems, Joint Mathematics Meetings, Online due to COVID-19, Jan 5, 2021. 

\item\emph{Persistent homology of complex networks for dynamic state detection in time series}. Topological Data Analysis and Beyond Workshop, Neurips 2020, Online due to COVID-19, Dec 11, 2020. 

\item\emph{The Truncated Interleaving Distance for Reeb Graphs}. TGDA Seminar, Ohio State University, Online due to COVID-19, Nov 17, 2020. 

\item\emph{\href{https://mediaspace.msu.edu/media/t/1_yghdjfiz}{Combining network analysis and persistent homology for classifying behavior of time series}}. CMSE Brownbag Seminar, MSU, Nov 13, 2020. 

\item\emph{Combining network analysis and persistent homology for classifying behavior of time series}. Virtual Mathematics Colloquium, Clarkson University, Online due to COVID-19, Oct 19, 2020. 

\item\emph{Combining network analysis and persistent homology for classifying behavior of time series}. Second Symposium on Machine Learning and Dynamical Systems, Fields Institute, Online due to COVID-19, Sep 29, 2020. 

\item\emph{\href{https://youtu.be/nyO4KQTmwzU}{The Truncated Interleaving Distance for Reeb Graphs}}. GEOTOP-A Online Seminar, Sep 18, 2020. 

\item\emph{Convergence of Mapper Graphs}. MSU TDA Seminar, East Lansing, MI, Aug 6, 2020. 

\item\emph{Featurization of Persistence Diagrams using Template Functions for Machine Learning Tasks}. Diff CVML - CVPR Keynote, Online due to COVID-19., June 29, 2020. 

\item\emph{\href{https://youtu.be/2mscXkz_o4A}{Topological Data Analysis for Quantifying Plant Morphology}}. SIAM MDS, Online due to COVID-19., June 2, 2020. 

\item\emph{Drawing Time-Varying Reeb Graphs}. MSU AMS/AWM Grad Student Seminar, Mar 9, 2020. 

\item\emph{Applications of TDA: From Reeb Graphs to Diagrams, From Time Series to Plant Morphology}. MSU TDA Seminar, East Lansing, MI, Mar 9, 2020. 

\item\emph{\href{https://youtu.be/Moh9lF4UVs4}{Featurization of Persistence Diagrams using Template Functions for Machine Learning Tasks}}. Applied Algebraic Topology Research Network (AATRN), Online Seminar, Jan 29, 2020. 

\item\emph{Featurization of Persistence Diagrams using Template Functions for Machine Learning Tasks}. Joint Math Meetings (JMM), Denver, CO, Jan 17, 2020. 

\item\emph{Featurization of Persistence Diagrams using Template Functions for Machine Learning Tasks}. Canadian Mathematical Society Winter Meeting, Toronto, Canada, Dec 7, 2019. 

\item\emph{Measuring the Shape of Data: The Shape of Hurricanes}. Big Data Colloquium, Grand Valley State University, Allendale, MI, Nov 22, 2019. 

\item\emph{Featurization of Persistence Diagrams for Classifying Attractors}. SIAM Dynamical Systems, Snowbird, Utah, May 22, 2019. 

\item\emph{Featurization of persistence diagrams using template functions for machine learning tasks}. SIAM Great Lakes Sectional Meeting, University of Michigan, Ann Arbor, MI, April 27, 2019. 

\item\emph{Utilizing persistent homology for classifying behavior of time series}. Workshop on Data Driven Dynamics: Algebraic Topology, Combinatorics and Analysis, CRM, Montreal, Canada, April 18, 2019. 

\item\emph{The Interleaving Distance for a Category with a Flow}. MSU AMS/AWM Grad Student Seminar, Feb 7, 2019. 

\item\emph{The Interleaving Distance for a Category with a Flow}. Upstate New York Topology Seminar, Albany, NY, Nov 10, 2018. 

\item\emph{Quantum Computation in a Topological Data Analysis Pipeline}. 2018 D-Wave Qubits North America, Knoxville, TN, Sep 25, 2018. 

\item\emph{Interleavings for categories with a flow and the hom-tree lower bound}. BIRS-CMO Workshop on Multiparameter Persistent Homology, Oaxaca, Mexico, Aug 9, 2018. 

\item\emph{Topological Data Analysis}. Summer Undergraduate Research Institute in Experimental Mathematics (SURIEM), MSU, East Lansing, MI, June 21, 2018. 

\item\emph{Topological Data Analysis}. iCER ACRES REU, MSU, East Lansing, MI, June 13, 2018. 

\item\emph{Topological Data Analysis for Time Series Analysis}. Abel Symposium, Geiranger, Norway, June 6, 2018. 

\item\emph{Reeb graphs, Mapper graphs, and Metrics}. IMA Special Workshop: Bridging Statistics and Sheaves, Minneapolis, MN, May 21, 2018. 

\item\emph{Approximating Continuous Functions on Persistence Diagrams for Machine Learning Tasks}. TRIPODS Seminar: Geometry and Topology of Data, ICERM, Brown University, Providence, RI, December 13, 2017. 

\item\emph{Quantifying and Comparing Shape in Data}. Colloquium, Dept of Mathematics, University of Michigan at Dearborn, Dearborn, MI, December 6, 2017. 

\item\emph{Approximating Continuous Functions on Persistence Diagrams for Machine Learning Tasks}. Geometry and Topology Seminar, Math Dept, MSU, East Lansing, MI, November 16, 2017. 

\item\emph{What is Topological Data Analysis}. CMSE Brown-bag Lecture Series, MSU, East Lansing, MI, October 17, 2017. 

\item\emph{Applications of Persistence to Time Series Analysis}. SIAM Central States Section 2017 Meeting, Colorado State University, Fort Collins, CO, September 30, 2017. 

\item\emph{Introduction to Categorical Approaches in Topological Data Analysis II}. Topology, Computation and Data Analysis, Schloss Dagstuhl, Wadern, Germany, July 17, 2017. 

\item\emph{Applications of Persistence to Time Series Analysis}. SIAM Conference on Dynamical Systems, Snowbird, Utah, May 23, 2017. 

\item\emph{Reeb graphs, Mapper graphs, and Metrics}. 47th Annual John H.~Barrett Memorial Lectures, University of Tennessee - Knoxville, TN, Mar 2, 2017. 

\item\emph{The Convergence of Mapper}. AWM Research Symposium, UCLA, Los Angeles, CA, Apr 8, 2017. 

\item\emph{The Convergence of Mapper}. Brown University, Providence, RI, Mar 9, 2017. 

\item\emph{A Topological Approach to Data Science}. Dept.~of Mathematics, Montana State University, Bozeman, MT, Feb 13, 2017. 

\item\emph{A Topological Approach to Data Science}. Dept.~of Computational Mathematics, Science, and Engineering, Michigan State University, East Lansing, MI, Jan 11, 2017. 

\item\emph{The interleaving distance for posets}. Joint Mathematics Meetings, Atlanta, GA, Jan 4, 2017. 

\item\emph{The Convergence of Mapper}. New York Applied Topology Meetings, Columbia University, New York, NY, Dec 9, 2016. 

\item\emph{The interleaving distance for posets}. Union College Mathematics Conference, Schenectady, NY, Dec 3, 2016. 

\item\emph{The Reeb graph interleaving distance}. Computer Science Seminar, St.~Louis University, Oct 12, 2016. 

\item\emph{Utilizing Topological Data Analysis to Detect Periodicity}. International Workshop on Topological Data Analysis in Biomedicine at ACM-BCB, Seattle, WA, Oct 2, 2016. 

\item\emph{The interleaving distance}. Geometry and Topology Seminar, Math Department, University at Buffalo, Buffalo, NY, Sep 23, 2016. 

\item\emph{Applications of Persistence to Time Series Analysis}. Topology, Geometry, and Data Analysis Conference (TGDA), Ohio State University, Columbus, OH, May 20, 2016. 

\item\emph{Topological Data Analysis}. Junior STEM Idea Exchange (JUSIE): Lightning event, Albany, NY, May 10, 2016. 

\item\emph{What Does it Mean for Data to Have Shape?}. Workshop on the Shape of Educational Data, Fairfax, VA, April 7, 2016. 

\item\emph{The Reeb Graph Interleaving Distance and its Application to Data Analysis}. Combinatorics Seminar, SUNY Binghamton, Binghamton, NY, Mar 15, 2016. 

\item\emph{Topological Data Analysis and its Application to Atmospheric Science Data}. Dept of Atmospheric and Environmental Sciences Colloquium, UAlbany, Albany, NY, Feb 15, 2016. 

\item\emph{Topological Data Analysis}. Math Department Student Seminar, Union College, Schenectady, NY, Jan 19, 2016. 

\item\emph{Applied Category Theory and the Reeb Graph Interleaving Distance}. Geometry and Topology Seminar, NCSU, Raleigh, NC, Jan 13, 2016. 

\item\emph{Complexity of the Reeb Graph Interleaving Distance}. CS Theory Seminar, NCSU, Raleigh, NC, Jan 11, 2016. 

\item\emph{Reeb Graph Approximation with Guarantees}. Joint Mathematics Meetings, Seattle, WA, Jan 9, 2016. 

\item\emph{Reeb Graph Approximation with Guarantees}. Fall Workshop on Computational Geometry (FWCG), Buffalo, NY, Oct 23, 2015. 

\item\emph{The Reeb Graph Interleaving Distance }. UAlbany - Algebra/Topology Seminar, Sep 3 (Part I) and Sep 10 (Part II), 2015. 

\item\emph{A New Metric for $\mathbb{R}$-graph Comparison}. AFRL Annual Applied Topology Workshop, Rome, NY, Aug 7, 2015. 

\item\emph{Demo: Using Perseus to Compute Persistent Homology}. UAlbany Applied Topology Reading Group, Aug 6, 2015. 

\item\emph{The Interleaving Distance}. CG Week, Symposium on Computational Geometry (SoCG), Eindhoven, Netherlands, June 25, 2015. 

\item\emph{Strong Equivalence of the Interleaving and Functional Distortion Metrics for Reeb Graphs}. Symposium on Computational Geometry (SoCG), Eindhoven, Netherlands, June 24, 2015. 

\item\emph{Using Topology to Understand Big Data}. Junior STEM Idea Exchange (JUSIE), University at Albany - SUNY, Mar 25, 2015. 

\item\emph{Strong Equivalence of Reeb Graph Metrics}. Invited Lecture, TU Munich, Munich, Germany, Mar 17, 2015. 

\item\emph{The Cosheaf-Less Reeb Graph Interleaving Distance}. Seminar on Computational Geometry, Schloss Dagstuhl, Wadern, Germany, Mar 12, 2015. 

\item\emph{The Interleaving Distance for Reeb Graphs}. Applied Algebraic Topology Research Network - Online Seminar, Feb 25, 2015. 

\item\emph{Using Topology to Understand Big Data}. Women in Science and Engineering (WISH) Lunch, University at Albany, Albany, NY, Jan 15, 2015. 

\item\emph{Towards Predicting and Preventing Machine Chatter Using Persistent Homology}. Invited lecture, Workshop on Topology : Identifying Order in Complex System, Rutgers, New Brunswick, NJ, Oct 18, 2014. 

\item\emph{Using Topology to Understand Big Data}. Invited lecture, Modern Math Workshop, Los Angeles, CA, Oct 16, 2014. 

\item\emph{Interleavings of Reeb Graphs}. Invited Lecture, AIM Workshop on Generalized Persistence and Applications, Sept 18, 2014. 

\item\emph{Interleavings of Reeb Graphs}. Invited Lecture, ATMCS 6, Vancouver, Canada, May 29, 2014. 

\item\emph{Topological Data Analysis :: Persistent Homology and Applications}. Invited Lecture, Systems Information Learning Optimization (SILO), University of Wisconsin - Madison, Madison, WI, April 23, 2014. 

\item\emph{A Distance Measure on Reeb Graphs}. Invited Lecture, Topology, Geometry and Data Seminar, Ohio State University, Columbus, OH, April 11, 2014. 

\item\emph{Using Persistence to Explore Equilibria of Delay Equations}. Invited Lecture, Spring Topology and Dynamics Conference, University of Richmond, Richmond, VA, March 14, 2014. 

\item\emph{Categorification of Reeb Graphs}. Invited Lecture, Workshop on Topological Systems : Communication, Sensing, and Actuation, IMA, Minneapolis, MN, March 6, 2014. 

\item\emph{Categorification of Reeb Graphs}. SAMSI Workshop, LDHD: Topological Data Analysis, Durham, NC, February 5, 2014. 

\item\emph{Extending Statistical Methods to Computational Topology}. SIAM Minisymposium on Applied and Computational Geometry, JMM, Baltimore, MD, January 17, 2014. 

\item\emph{A Statistical Approach for Improving Topological Data Analysis}. University of Rochester, Rochester, NY, January 10, 2014. 

\item\emph{Categorification of Reeb Graphs}. Topology Seminar, Tulane University, New Orleans, LA, November 5, 2013. 

\item\emph{Categorification of Reeb Graphs}. IMA Postdoc Seminar, University of Minnesota, Minneapolis, MN, October 22, 2013. 

\item\emph{A Continuous Mean for Finite Sets of Persistence Diagrams}. Invited Lecture, Workshop: Topological Data Analysis, IMA, University of Minnesota, Minneapolis, MN, October 10, 2013. 

\item\emph{A Continuous Mean for Distributions of Persistence Diagrams}. Invited Lecture, SIAM Conference on Applied Algebraic Geometry, Fort Collins, CO, August 2, 2013. 

\item\emph{An Introduction to Topological Data Analysis}. Invited Lecture, SUNYIT, Utica, NY, June 11, 2013. 

\item\emph{Applications of Persistent Homology to Time Varying Systems}. PhD Defense, Duke University, March 28, 2013. 

\item\emph{Using Persistent Homology to Analyze Behavior in Dynamic Point Clouds}. SIAM Student Chapter Lecture Series, Colorado State University, Fort Collins, CO, November 5, 2012. 

\item\emph{Using Persistent Homology to Analyze Dynamic Point Clouds}. Data Research Training Grant Seminar, Duke University, Durham, NC, October 22, 2012. 

\item\emph{An Intro to Persistent Homology and Some Applications}. Graduate/Faculty Seminar, Duke University, Durham, NC, September 7, 2012. 

\item\emph{Metrics on Vineyards}. Computational Geometry Week, Chapel Hill, NC, June 20, 2012. 

\item\emph{Primates and Vineyards}. Duke University Math Slam, Duke University, Durham, NC, March 23, 2012. 

\item\emph{Utilizing Ideas from Persistent Homology to Compute Probabilistic Sensor Network Coverage}. Algebra/Combinatorics Seminar, NC State University, Raleigh,NC, January 13, 2012. 

\item\emph{Applied Topology: Basic Ideas and a Mess of Applications}. Graduate/Faculty Seminar, Duke University, Durham, NC, October 14, 2011. 

\item\emph{Using Persistent Homology to Compute Probabalistic Failure of a Sensor Network}. SIAM Conference on Applied Algebraic Geometry, Raleigh, NC, October 6, 2011. 

\item\emph{Computing Probabilistic Sensor Network Coverage via Algorithms Utilizing Persistent Homology}. Geometry and Topology Reading Group, Institute of Science and Technology, Klosterneuburg, Austria, June 20, 2011. 

\item\emph{Computing Probabilistic Sensor Network Coverage via Algorithms Utilizing Persistent Homology}. Computational Geometry Seminar, UNC Chapel Hill, Chapel Hill, NC, February 3, 2011. 

\item\emph{Failure Filtrations and Coverage of Fenced Sensor Networks}. Graduate/Faculty Seminar, Duke University, November 19, 2010. 

\end{etaremune}
\end{rSection}


%----------------------------------------------------------------------------------------
%
%----------------------------------------------------------------------------------------

\newpage
\begin{rSection}{Teaching Experience}

\begin{tabular}{cLccc}
\textbf{Course} & \textbf{Title} & \textbf{Institution} & \textbf{Semesters}& \textbf{Eval.}\\
% \color{gray}{\rule{1em}{\textwidth}}
\arrayrulecolor{Gray}\hline
CMSE 201 & Intro to Computational Modeling and Data Analysis & MSU & Spring, 2018
 & 1.81*
\\
\hline
CMSE 381 & Fundamentals of Data Science Methods & MSU & Spring, 2022
 & 
\\
\hline
CMSE 801 & Intro to Computational Modeling & MSU & Spring, 2020
 & 1.36*
\\
\hline
CMSE 802 & Methods Computational Modeling & MSU & Spring, 2021
 & 1.12*
\\
\hline
CMSE 491/890 & Topological Analysis of Large Datasets & MSU & \begin{tabular}{@{}c@{}} Fall, 2017\\Fall, 2021\end{tabular} & \begin{tabular}{@{}c@{}} 1.75/1.18*\\\phantom{x}\end{tabular}\\
\hline
MTH 132 & Calculus I & MSU & Fall, 2020
 & 1.52*\textsuperscript{\textdagger}
\\
\hline
MTH 481 & Discrete Mathematics I & MSU & Fall, 2018
 & 1.42*
\\
\hline
AMAT 840 & Applied Topology & UAlbany & Spring, 2015
 & 4.50
\\
\hline
AMAT 587 & Graph Theory & UAlbany & Fall, 2015
 & 5.00
\\
\hline
AMAT 540B & Topology II & UAlbany & Spring, 2017
 & 4.60
\\
\hline
AMAT 502 & Modern Computing for Mathematicians & UAlbany & Spring, 2016
 & 5.00
\\
\hline
AMAT 363 & Statistics & UAlbany & \begin{tabular}{@{}c@{}} Fall, 2014\\Fall, 2015\\Spring, 2016\\Spring, 2017\end{tabular} & \begin{tabular}{@{}c@{}} 4.96\\4.87\\4.84\\4.67\end{tabular}\\
\hline
TIP & Mobius Strips, Klein Bottles, and Fractals: The Mathematics of Distortion & Duke TIP & July, 2012
 & 4.82
\\
\hline
Math 32L & Laboratory Calculus II & Duke & Spring, 2011
 & 4.00
\\
\hline
Math 31L & Laboratory Calculus I & Duke & Fall, 2009
 & 3.00
\\
\hline
Math 25L & Laboratory Calculus and Functions I & Duke & Fall, 2008
 & 
\\
\hline
% × & × & × & ×\\
% × & × & × & ×
\end{tabular}

*Note that while all evaluation scores are provided out of 5, MSU defines a score of 1 to be the best, while all other scores are based on 5 as the best.
Those evaluations using a best score of 1 are marked with an asterisk.

\textdagger Aggregated score.  Due to Covid-19, course was taught online, with 337 official students of record across 13 sections.


% Listed evaluation scores based on
% \textit{Composite Profile Factors: Instructor Involvement} for CMSE courses and
% the \textit{Question 15: What is your overall rating of this instructor?} for MTH 481.
% See \cref{sec-SIRS} for complete SIRS scores.

\end{rSection}


%----------------------------------------------------------------------------------------
%----------------------------------------------------------------------------------------

% \newpage
\begin{rSection}{Students Advised}


\textbf{\textit{Current Graduate Students}}
\begin{list}{}{\leftmargin=1em}
   \itemsep -0.1em %\vspace{-0.5em} % Compress items in list together for aesthetics






   \item \textbf{\href{https://www.egr.msu.edu/~amezqui3/aboutme.html}{Erik Am\'ezquita}}, PhD Student, MSU CMSE \hfill Sep 2018 - Present
\\
   \phantom{XXX} \textit{Jointly advised with Dan Chitwood}
   \item \textbf{\href{https://www.egr.msu.edu/~mcgui176/}{Sarah McGuire}}, PhD Student, MSU CMSE \hfill Sep 2019 - Present


   \item \textbf{Xinyi (Elena) Wang}, PhD Student, MSU CMSE \hfill Sep 2020 - Present

   \item \textbf{\href{https://sites.google.com/view/christopherpotvin/home}{Christopher Potvin}}, PhD Student, MSU Math \hfill Jan 2021 - Present

   \item \textbf{\href{https://sites.google.com/manhattan.edu/racheleroca/home}{Rachel Roca}}, PhD Student, MSU CMSE \hfill Sep 2021 - Present
\\
   \phantom{XXX} \textit{Jointly advised with Danny Caballero}


   \item \textbf{David Munoz}, PhD Student, MSU CMSE \hfill Jan 2022 - Present
   

   \item \textbf{Astrid Olave}, PhD Student, MSU CMSE \hfill Jan 2022 - Present
   

   \item \textbf{Valeri Jean-Pierre}, PhD Student, MSU Math \hfill Jan 2022 - Present
   
   \item \textbf{Ishika Ghosh}, PhD Student, MSU CMSE \hfill Aug 2022 - Present
   
   \item \textbf{Danielle Barnes}, PhD Student, MSU CMSE \hfill June 2022 - Present




\end{list}


\textbf{\textit{Postdocs}}
\begin{list}{}{\leftmargin=1em}
   \itemsep -0.1em %\vspace{-0.5em} % Compress items in list together for aesthetics


   \item \textbf{\href{http://sci.utah.edu/~sourabh/}{Sourabh Palande}}, Postdoc, MSU CMSE \hfill Sep 2020 - Present\\
   \phantom{XXX} \textit{Jointly advised with Dan Chitwood}

   \item \textbf{\href{https://sperciva.github.io/}{Sarah Percival}}, Postdoc, MSU BMB \hfill Sep 2021 - Present\\
   \phantom{XXX} \textit{Jointly advised with Dan Chitwood, Beronda Montgomery, Arjun Krishnan,}\\
   \phantom{XXX} \textit{and Aman Husbands (OSU)}


\end{list}

\textbf{\textit{Former Group Members}}
\begin{list}{}{\leftmargin=1em}
   \itemsep -0.1em %\vspace{-0.5em} % Compress items in list together for aesthetics


   \item \textbf{\href{https://sites.google.com/view/anastasiostefanou/home}{Anastasios Stefanou}}, PhD student, UAlbany Math\hfill Oct 2015 - Aug 2018
\\
   \phantom{XXX} \textit{Jointly advised with Justin Curry}
\\
   \phantom{XXX} \textit{First job post-PhD: Postdoc, Ohio State University}


   \item \textbf{\href{https://www.sarahtymochko.com/}{Sarah Tymochko}}, PhD Student, MSU CMSE \hfill Sept 2017 - May 2022
\\
   \phantom{XXX} \textit{First job post-PhD: Postdoc, UCLA}


   \item \textbf{\href{https://ismailguzel.github.io/}{\.{I}smail G\"{u}zel}}, Visiting Researcher, MSU CMSE \hfill Sep 2022 - Aug 2023

   \item \textbf{Christopher Sukhu}, MS Student, MSU CMSE \hfill May 2017 - May 2019

   \item \textbf{Mitchell Eithun}, MS Student, MSU CMSE \hfill Jan 2018 - Aug 2019
\\
   \phantom{XXX} \textit{Jointly advised with Dan Chitwood}




   \item \textbf{\href{www.shelleykandola.com}{Shelley Kandola}}, Postdoc, MSU Math \hfill Sep 2019 - July 2021
   % \phantom{XXX} \textit{Jointly advised with Dan Chitwood}

   \item \textbf{\href{https://research.tue.nl/en/persons/tim-ae-ophelders}{Tim Ophelders}}, Postdoc, MSU CMSE \hfill Sep 2018 - Aug 2020\\
   \phantom{XXX} \textit{Jointly advised with Dan Chitwood}
   \item \textbf{Michelle Quigley}, Postdoc, MSU Horticulture \hfill Sep 2018 - Sep 2020\\
   \phantom{XXX} \textit{Jointly advised with Dan Chitwood}
\end{list}

\textbf{\textit{Committee Membership}}
\begin{list}{}{\leftmargin=1em}
   \itemsep -0.1em %\vspace{-0.5em} % Compress items in list together for aesthetics


   % \item \textbf{Danielle Barnes}, PhD Student, Advisor: Jos\'e Perea, MSU CMSE\hfill Sept 2017 - Present



   % \item \textbf{Megan Kress}, Masters Student, Advisor: Andy Finley, MSU Forestry,\hfill Oct 2017 - Present

   \item \textbf{Nathan Brugnone}, PhD Student, Advisor: Robert Richardson, \hfill  2022 - Present\\
   \phantom{xxx}MSU CSUS/CMSE 
   \item \textbf{Chris St.~Clair}, PhD Student, Advisor: Matt Hedden, MSU Math\hfill Oct 2020 - Present
   \item \textbf{Alexander Harnisch}, PhD Student, Advisor: Claudio Kopper, \hfill March 2021 - Present\\
   \phantom{xxx} MSU PA/CMSE
   \item \textbf{Michael Quail}, PhD Student, Advisor: Kristen Bieda, \hfill Nov 2021 - Present\\
   \phantom{xxx} MSU Program in Mathematics Education
    \item \textbf{Brian Bollen}, PhD Student, Advisor: Josh Levine, \hfill Graduated May 2022\\
    \phantom{xxx}Arizona State University Program in Applied Mathematics
   \item \textbf{Rachel Domagalski}, PhD Student, Advisor: Bruce Sagan, MSU Math\hfill Graduated Aug 2021
   \item \textbf{Nick Young}, PhD Student, Advisor: Danny Caballero, \hfill Graduated Aug 2021\\
   \phantom{xxx}MSU Physics/CMSE

   \item \textbf{Zixuan Cang}, PhD Student, Advisor: Guowei Wei, MSU Math\hfill Graduated July 2018

   \item \textbf{Hitesh Gakhar}, PhD Student, Advisor: Jos\'e Perea, MSU Math\hfill Graduated May 2020

   \item \textbf{Luis Polanco}, PhD Student, Advisor: Jos\'e Perea, MSU CMSE/Math\hfill Graduated May 2022
   % \item \textbf{Matthew Piekenbrock}, PhD Student, Advisor: Jos\'e Perea, MSU CMSE\hfill Sept 2019 - Present

   \item \textbf{Audun Myers}, PhD Student, Advisor: Firas Khasawneh, MSU ME\hfill Graduated May 2022
   \item \textbf{Melih Yesili}, PhD Student, Advisor: Firas Khasawneh, MSU ME\hfill Graduated May 2022
\end{list}

% \vspace{1em}

\textbf{\textit{Undergraduate and High School Students}}
\begin{list}{}{\leftmargin=1em}
   \itemsep -0.1em %\vspace{-0.5em} % Compress items in list together for aesthetics
   \item \textbf{Yash Gautam}, Undergraduate Student, Professorial Assistant, MSU\hfill Sep 2020 - Present

   \item \textbf{Vee Kalkunte and Sean Bergen}, Undergraduate Students, ACRES REU, MSU\hfill Summer 2021

   \item \textbf{Levent Batakci, Abby Branson, Bryan Castillo, Candace Todd}\hfill Summer 2020\\
   SURIEM REU, MSU

   \item \textbf{Kayla Makela}, Undergraduate Student, MSU\hfill Jan 2018 - Aug 2020

   \item \textbf{Joseph Sigler}, Undergraduate Student, Professorial Assistant, MSU\hfill Sep 2017 - May 2019

   \item \textbf{Monika Francsics}, Undergraduate Student, MSU\hfill Summer 2018

   \item \textbf{Brian Bollen}, Undergraduate Student, UAlbany \hfill May 2016 - Aug 2017
   % Project title: \textit{Characterizing Chaos using Persistent Homology}

   \item \textbf{Akanksha Atrey}, Undergraduate Student, UAlbany \hfill Jan 2015 - June 2016
   % Project title: \textit{Reeb Graph Based Validation of Statistical Predictive Models for the Spread of AIDS}

   \item \textbf{Bill Dong}, High School Student, UAlbany \hfill Jun 2015 - Aug 2016
   % Project title: \textit{Using Persistent Homology to Quantify a Diurnal Cycle in Hurricanes}
\end{list}

\end{rSection}


%----------------------------------------------------------------------------------------
%
%----------------------------------------------------------------------------------------
% \newpage
\begin{rSection}{Service}



\textbf{Departmental and University Service}
\begin{list}{}{\leftmargin=1em}
   \itemsep -0.6em \vspace{-0.5em} % Compress items in list together for aesthetics

\item
MSU TDA Seminar, Creator and Co-Organizer with Shelley Kandola  \hfill Mar 2020 - Present

\item
CMSE Advisory Committee (AdCom), MSU\\
\phantom{xxx} \textit{Secretary} \hfill  Sept 2017 - Aug 2018\\
\phantom{xxx} \textit{Member} \hfill Sept 2020 - present

\item CMSE Hiring Committees, MSU \\
\phantom{xxx} \textit{TT Data Science} \hfill  Sept 2021- Feb 2022\\
\phantom{xxx} \textit{1855 Data Science } \hfill Mar 2022 - present
\item Math Undergraduate Committee, MSU \\
\phantom{xxx} \textit{TT Data Science} \hfill  Sept 2021- Feb 2022\\

\item
CMSE Graduate Studies Committee, MSU \hfill  Aug 2022 - Present 
\item
Member of College of Engineering Graduate Studies Commitee, MSU \hfill Sept 2018 - Aug 2020
\item
Member of CMSE Awards Committee, MSU \hfill Sept 2019 - Aug 2020
\item
CNS Strategic Plan Integration Committee, MSU \hfill Jan 2020 - Dec 2020
\item
Panel on Academic Careers for Engineering Graduate Students, MSU \hfill Nov 26, 2019
\item Member, Executive Committee for PlantBio NSF-NRT Grant, MSU
\hfill Sept 2018 - Aug 2019

\item
Math Dept.~Graduate Committee, UAlbany \hfill Sept 2016 - Aug 2017
\item
Math Dept Representative, UAlbany Undergraduate Research  \hfill Mar 2016\\
\phantom{MMM}Information Session and Forum
\item
Faculty Speaker, UAlbany Math Department Graduation Ceremony \hfill May 2015
\item
{Organizer, UAlbany Reading Group on Applied Topology } \hfill Summer 2015
% \item
% {Organizer, Coffee for Women in Math with Jill Pipher, Invited Speaker } \hfill Apr 2015
\item
{UAlbany Math Dept Colloquium Committee} \hfill Sept 2014-Aug 2017
\item
{Organizer for the IMA Postdoc Seminar} \hfill Sept 2013-Aug 2014
\item
{President, Noetherian Ring of Duke University} \hfill Sept 2012-May 2013
\item
{Mathbio Seminar Organizer, Duke Math Dept} \hfill Spring 2012
\item
{Project Mentor: Duke Workshop on Applications of Math to}
\hfill May 16-20, 2011\\
\phantom{MMM}{Physiology and Medicine}

\item
{Tea Organizer for Duke Mathematics Dept} \hfill Sept 2010 - May 2011
% \item
% {Panelist for Duke Math Dept on Getting a job after your PhD} \hfill Mar 2013
% \item
% {Panelist for Duke Math Dept Recruitment Weekend Info Session} \hfill March 2010, March 2011
\item
{Graduate Student Representative for Duke Math Department}\hfill Jan 2009 - Dec 2009
\end{list}

\textbf{Professional Service}
\begin{list}{}{\leftmargin=1em}
   \itemsep -0.6em \vspace{-0.5em} % Compress items in list together for aesthetics
\item Organizer, \href{https://meetings.siam.org/sess/dsp_programsess.cfm?SESSIONCODE=71292}
{Workshop on Topological Signal Processing} at SIAM DS 21 \hfill May 2021
\item
Workshop Committee Member, \href{https://socg20.inf.ethz.ch/}{Symposium on Computational Geomtry (SoCG)} \hfill 2020
\item
Program Committee Member, \href{https://women-plus-datascience.github.io/}{Women+ Data Science Monthly Meetup}, MSU \hfill Fall 2020 % Note: footnote[2] gives the dagger
\item Mentor, \href{https://mathalliance.org/mentor/liz-munch/}{National Alliance for Doctoral Studies in the Mathematical Sciences} \hfill Fall 2020 - present
\item
Steering Committee Member, ATMCS \hfill Mar 2017 - present
\item
Steering Committee Member, \href{https://awmadvance.org/research-networks/wincomptop-women-in-computational-topology/}{Women in Computational Topology} \hfill Sep 2016 - present
\item
Organizer, \href{https://www.ima.umn.edu/2017-2018/SW5.21-25.18}{IMA Workshop: Bridging Statistics and Sheaves} \hfill May 2018
\item
{Organizer for Special Session on Sheaves in TDA at the Joint Math Meetings} \hfill Jan 2017
\item
{Organizer, The 4th Annual Minisymposium on Computational Topology,  } \hfill Jun 2015\\
\phantom{MMM}{CG Week at the Symposium on Computational Geometry}
\item
{Session Organizer, AWM Research Symposium} \hfill Apr 2015
\item
{Organizer for Special Session at the Joint Math Meetings} \hfill Jan 2013
\end{list}


\textbf{Community Outreach}
\begin{list}{}{\leftmargin=1em}
   \itemsep -0.6em \vspace{-0.5em} % Compress items in list together for aesthetics
\item
Panelist on Careers in Academia, Women in Science Conference, Notre Dame \hfill Oct 2018
\item Interview for The Girls' Angle Bulletin,  ISSN 2151-5700 \hfill Oct/Nov 2015
\item
{NYS Master Teacher Program Application Review Committee} \hfill Sept 2014
\item
{Volunteer for FEMMES  Capstone Events} \hfill April 2011, April 2012
\end{list}

\textbf{Editorial Boards \& Conference Program Committees}

\begin{list}{}{\leftmargin=1em}
   \itemsep -0.6em \vspace{-0.5em} % Compress items in list together for aesthetics
   \item Inaugural Editorial Board Member, Computing in Geometry and Topology (CGT)  \hfill 2021-Present
\item
Program Committee Member, Symposium on Computational Geometry (SoCG) \hfill 2023
\item
Program Committee Member, ``Applications of Topological Data Analysis \hfill 2021\\
\phantom{MMM}to Big Data" Workshop, IEEE Big Data Conference
\item
Program Committee Member, \href{https://www.renyi.hu/conferences/socg18/}{Symposium on Computational Geometry (SoCG)} \hfill 2018
\item {Program Committee Member, Fall Workshop on Computational Geometry} \hfill 2014, 2016
\item
{Scientific Committee,
ATMCS7, Torino, Italy} \hfill Jul 2016\\
 \phantom{MMM}{Applied Topology: Methods, Computation, and Science}
\end{list}




\textbf{Review and Referee}

\textit{Journals}
\begin{list}{}{\leftmargin=1em}
   \itemsep -0.6em \vspace{-0.5em} % Compress items in list together for aesthetics
   \item AMS Notices \hfill 2022
   \item Chaos \hfill 2021
   \item Climate Dynamics \hfill 2021
   \item Computational Geometry: Theory and Applications \hfill 2016
   \item Discrete and Computational Geometry \hfill 2015, 2019
   \item Image and Vision Computing \hfill 2020
   \item Medical Image Analysis \hfill 2020
   \item Journal of Applied and Computational Topology (JACT) \hfill
   2018-2021
   % 2019, %x2
   % 2020
   \item Journal of Computational Geometry (JoCG) \hfill 2018, 2019, 2021
   \item Journal of Machine Learning Research (JMLR) \hfill 2017
   \item Patterns \hfill 2021
   \item PLOS ONE \hfill 2015, 2016
   \item PNAS \hfill 2022
   \item Science Advances \hfill 2021
   \item SIAM Journal on Applied Algebra and Geometry (SIAGA) \hfill 2017
   \item SIAM Journal on Mathematics of Data Science (SIMODS) \hfill 2022
   \item SIAM Review \hfill 2021
   \item IEEE Transactions on Pattern Analysis and Machine Intelligence (TPAMI) \hfill 2019, 2020
   \item WIRES Computational Statistics \hfill 2019
%    \item Fall Workshop on Computational Geometry \hfill 2014
\end{list}
\textit{Conferences}
\begin{list}{}{\leftmargin=1em}
   \itemsep -0.6em \vspace{-0.5em} % Compress items in list together for aesthetics
   \item International Symposium on Algorithms and Computation (ISAAC) \hfill 2017
   \item Symposium on Computational Geometry (SoCG) \hfill 2014, 2016, 2018, 2019, 2020, 2021
   \item Symposium on Discrete Algorithms (SoDA) \hfill 2017, 2020
   \item International Conference on Robotics and Automation (ICRA) \hfill 2019
   \item IEEE Scientific Visualization (SciVis) \hfill 2019
\end{list}
\textit{Funding agencies}
\begin{list}{}{\leftmargin=1em}
   \itemsep -0.6em \vspace{-0.5em} % Compress items in list together for aesthetics
   \item NSF: CISE  \hfill 2016, 2018, 2019, 2021, 2022
   \item NSF: DMS \hfill 2018, 2019
   \item AFOSR \hfill 2022
\end{list}
\textit{Other}
\begin{list}{}{\leftmargin=1em}
   \itemsep -0.6em \vspace{-0.5em} % Compress items in list together for aesthetics
   \item MathSciNet \hfill 2018-Present
   \item zbMath \hfill 2021-Present
\end{list}

\end{rSection}
%----------------------------------------------------------------------------------------
%	Professional Affiliations
%----------------------------------------------------------------------------------------
% \newpage
\begin{rSection}{Professional Affiliations}

\begin{list}{}{\leftmargin=0em}
   \itemsep -0.6em %\vspace{-0.5em} % Compress items in list together for aesthetics
 \item American Mathematical Society (AMS)
 \item Society for Industrial and Applied Mathematics (SIAM)
 \item Association for Women in Mathematics (AWM)
 \item Association for Computing Machinery (ACM)
 \item \href{https://mathalliance.org/mentor/liz-munch/}{Math Alliance Mentor}
 \item National Association of Mathematicians (NAM)
 \item American Harp Society (AHS)
\end{list}


\end{rSection}
%----------------------------------------------------------------------------------------
%	TECHNICAL STRENGTHS SECTION
%----------------------------------------------------------------------------------------
\begin{rSection}{Technical Strengths}

\begin{tabular}{ @{} >{\bfseries}l @{\hspace{6ex}} l }
Software and Coding & Python, MATLAB, R, \LaTeX, Inkscape, HTML, CSS\\
Operating Systems & Linux, Windows, Android
\end{tabular}

\end{rSection}

%----------------------------------------------------------------------------------------
%	EXAMPLE SECTION
%----------------------------------------------------------------------------------------

%\begin{rSection}{Section Name}

%Section content\ldots

%\end{rSection}

%----------------------------------------------------------------------------------------

\end{document}
